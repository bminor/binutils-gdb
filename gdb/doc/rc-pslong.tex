% $Id$
% To choose PS fonts (long PS fontnames) for the refcard, link
% or copy this file to rcfonts.tex
% 
%The Times-Roman family is both more attractive and more compact than 
%Computer Modern.  On the other hand, while common, it is not free.  
%There are three sets of font definitions:
%  1) rc-cm.tex uses the free (Computer Modern) fonts
%  2) rc-ps.tex uses common PostScript fonts, with fontnames from the
%  Karl Berry scheme recommended in the documentation for dvips.
%  3) rc-pslong.tex uses common PostScript fonts, with the long names
%  used by PostScript programs directly.
%
% One caution: due to differing character ordering between TeX and PS,
%if your TeX is pre-3.0, or if you don't have virtual Courier
%matching the TeX character positions, you might want to use CMtt for
%\tt even if you switch to PostScript fonts for the rest of the text.
%
%-------------------- PostScript fonts (long names) -----------------
\font\bbf=Times-Bold at 10pt
\font\vbbf=Times-Bold at 12pt
\font\smrm=Times-Roman at 6pt
\font\brm=Times-Roman at 10pt
\font\rm=Times-Roman at 8pt
\font\it=Times-Italic at 8pt
\font\tt=Courier at 8pt
% Used only for \copyright, replacing plain TeX macro.
\font\sym=Symbol at 7pt
\def\copyright{{\sym\char'323}}
%-------------------- end font defs ---------------------------------
